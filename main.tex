\documentclass[12pt,a4paper]{article}


\usepackage{import}
\usepackage{myheaderstuff}

\usepackage[style=authoryear,date=year]{biblatex}



\addbibresource{syllabus.bib}
%%%%%%%%%%%%%%%%%%%%%%%%%%%%%%%%%%%%
\begin{document}
\begin{center}
{\Large \textsc{478981-HS2025-0-\\
\medskip
Energy and Mobility Policies for Decarbonisation}} \\
\bigskip
{\large
Institut für Politikwissenschaft \& \\
Oeschger-Zentrum für Klimaforschung (OCCR) \\
Universität Bern\\
}
\end{center}


\begin{center}
\rule{\textwidth}{0.4pt}
\begin{minipage}[t]{\textwidth}
\medskip
\begin{tabular}{ll} \medskip
\textbf{Leitung} & Dr. G. Brückmann  \\  \medskip
\textbf{Kontakt} &  \myemail{gracia.brueckmann@unibe.ch}{[PolDec25]}{gracia.brueckmann@unibe.ch} \\  \medskip
\textbf{Sprechstunde} & nach Buchung via \href{https://cal.com/brueckmann/sprechstunde}{cal.com/brueckmann/sprechstunde} \\  \medskip
\textbf{Büro} & Institutsgebäude vonRoll A 169; Austausch auch per \href{https://unibe-ch.zoom.us/my/graciabrueckmann}{Zoom} möglich  \\ 
\end{tabular}
\end{minipage}
\rule{\textwidth}{0.2pt} 
\begin{minipage}[t]{\textwidth} \smallskip
\begin{tabular}{ll} \smallskip
\textbf{Seminarzeit} & Mittwoch, 10:15-11:55 Uhr  \\  \medskip
\textbf{Seminarraum} &  Institutsgebäude vonRoll Seminarraum B 201  \\  \medskip
\textbf{Zielgruppe} & Master Politikwissenschaft \& Climate Sciences  \\  \medskip
\textbf{ECTS} &  6  \\ 
\end{tabular} \medskip
\end{minipage}
\rule{\textwidth}{0.4pt}
\end{center}
\medskip
\setlength{\unitlength}{1in}
\renewcommand{\arraystretch}{2}


\noindent\textbf{Kursbeschreibung} 
Angesichts der Klimakrise müssen alle Bereiche des täglichen Lebens bald dekarbonisiert werden. Insbesondere in Mobilitäts- und Energiesystemen sind Kohlenstoff-\textit{Lock-ins}, ein Hindernis, welches durch Politiken überkommen werden muss. Allerdings benötigen diese Politiken in Demokratien öffentliche Unterstützung, welche nicht immer gegeben ist.\\ 
Während der Seminarsitzungen üben die Studierenden in Diskussionen, eigene wissenschaftliche Einschätzungen zu veröffentlichten Studien zu artikulieren und zu verteidigen (\textit{Journal Club Format}). Daraus erarbeiten die Studierenden Ideen für die eigenen Forschungsarbeit zum Seminarthema, dem Konzept des \textit{Forschenden Lernens} entsprechend.\\
Die Studierenden erarbeiten ausserdem kurze Videos in denen \textit{Policy Makers} Forschungsergebnisse handlungsanleitend vorgestellt werden. Dies schult die Fähigkeiten in der Präsentation und Wissenschaftskommunikation und entspricht Forschungsergebnissen, die die wichtige Rolle von Storytelling aufzeigen (\cite{Graeber2024,Zabel2025}).\\ 

\medskip
\noindent\textbf{Voraussetzung} Die Teilnahme an diesem Seminar erfordert sowohl

\begin{itemize}
\item Kenntnisse von quantitativen, empirischen sozialwissenschaftlicher Forschungsmethoden (Experimentaldesign, Umfragedesign, quasi-experimentelle Methoden) als auch
    \item aktive Kenntnisse in der Anwendung von statistischer Analysesoftware (\href{https://www.r-project.org/}{R}, \href{https://www.python.org/}{Python} oder \href{https://www.stata.com/}{Stata}).
\end{itemize}
% Bei individuellen Fragen hierzu stehe ich vor Seminarstart gerne per Mail \myemail{gracia.brueckmann@unibe.ch}{[PolDec25] Abklärung Teilnahmevorausetzungen}{Email} zur Verfügung. Bitte kontaktieren Sie mich mit einer Übersicht bisher belegter Lehrveranstaltungen (Bachelor \& Master) und von weiteren Kenntnissen.\\


\medskip
\noindent\textbf{Lernziele}
Studierende werden durch dieses Seminar befähigt, die wichtigsten Herausforderungen bei der Umsetzung von Politiken zur Dekarbonisierung im Energie- und Mobilitätssektor zu verstehen, sowie politische Lösungsvorschläge zu identifizieren und zu beschreiben. Sie werden in der Lage sein, diese Ergebnisse in schriftlicher und mündlicher Form für ein wissenschaftliches, politisches oder allgemeines Publikum wirksam zu kommunizieren. Die Fähigkeiten in Gruppen zusammen zuarbeiten und Medienkompetenzen werden geschult.\\
Die Lehrveranstaltung soll die Fähigkeit der Studierenden zur kritischen Reflexion und Diskussion veröffentlichter, peer-begutachteter Arbeiten verbessern, indem die Studierenden regelmässig Fachartikel zum Thema Dekarbonisierungspolitiken lesen, analysieren und diskutieren. Dies zielt darauf ab, die Fähigkeiten zur kritischen Bewertung von Stärken und Schwächen der in der Literatur verwendeten Theorien und Forschungsmethoden zu entwickeln.\\
Die Erweiterung von Studien auf der Grundlage der eigenen Kritik der Studierenden an ausgewählten Arbeiten zielt auf eine umfassende Forschungserfahrung ab. Die Studierenden werden ihre Kenntnisse im Umgang mit statistischer Software zur Datenanalyse vertiefen, wenn die eigene Forschungsarbeit durchgeführt wird. Sie werden auch in die Lage versetzt, die Ergebnisse ihrer Analysen und von anderen Studien zu interpretieren und zu präsentieren und daraus korrekte Schlussfolgerungen und politische Implikationen abzuleiten.\\
Überdies soll der Kurs das Verständnis für die Bedeutung von Replikation und offener Wissenschaft bei den Teilnehmenden fördern.



\medskip
\noindent\textbf{Ilias} 
 \begin{itemize}

\item \textbf{Alle} Kursinhalte, Materialien und \textbf{Ankündigungen}, 
 werden im \href{https://ilias.unibe.ch/goto_ilias3_unibe_crs_3414477.html}{Ilias} zur Verfügung gestellt. 
%\item Dort befindet sich auch ein Forum, in dem sich die Teilnehmenden unter sich austauschen können. 
\item  Bitte besuchen Sie den \href{https://ilias.unibe.ch/goto_ilias3_unibe_crs_3414477.html}{Ilias}-Kurs regelmässig unter \url{https://ilias.unibe.ch/goto_ilias3_unibe_crs_3414477.html}. 
 \end{itemize}



\medskip
\noindent \textbf{Vorläufiger Semesterplan (Stand: \today)}
\begin{itemize}
    \item Der untenstehende Zeitplan ist vorläufig. Änderungen werden im Seminar und auf \href{https://ilias.unibe.ch/goto_ilias3_unibe_crs_3414477.html}{Ilias} bekannt gegeben.
    \item Nutzen Sie das Seminar für Fragen, anhand derer Sie und alle Teilnehmenden profitieren können. 
    \smallskip 
\item Die hier angegebenen Artikel (Pflichtlektüre) bitte vor der Sitzung komplett durcharbeiten. 
\end{itemize}
\newpage
%\medskip
\textsc{Sitzung 1 \dotfill Motivation \& Organisation} \smallskip \\ 
{\color{darkgreen}{\Rectangle}} Wieso dieses Seminar? Wieso Forschendes lernen? Austausch zu Erwartungen und Zielen für dieses Seminar basierend auf diesem Syllabus. \\
{\color{darkgreen}{\Rectangle}} 
\fullcite{Dubash2025}\\ 
\medskip  \\	


\textsc{Sitzung 2 \dotfill Die Sektoren Mobilität \& Energie}  \smallskip \\ 
{\color{darkgreen}{\Rectangle}} Warum die Sektoren Mobilität und Energie dekarbonisieren? Das Klima \&  andere Effekte\\
{\color{darkgreen}{\Rectangle}}  \fullcite{Sovacool2021} \\ 
{\color{darkgreen}{\Rectangle}} \fullcite{WatsonHartmann2025} \\
\medskip  \\	


\textsc{Sitzung 3 \dotfill Einführung Videoaufgabe} \smallskip \\ 
{\color{darkgreen}{\Rectangle}} Einführung in die Videoaufgabe; Warum eigentlich Experimente?\\ 
{\color{darkgreen}{\Rectangle}}  \fullcite{Pereira2024}\\ 
 {\color{darkgreen}{\Rectangle}} \fullcite{Sovacool2025} \\ 
 % {\color{darkgreen}{\Rectangle}} \fullcite{GrazerCharter2024} \\ 
%  {\color{darkgreen}{\Rectangle}} \fullcite{GrazerCharterDiskussion2025} \\  
\medskip  \\	


\textsc{Sitzung 4 \dotfill Braucht es Politiken?} \smallskip \\ 
{\color{darkgreen}{\Rectangle}} (Warum) braucht es Politiken oder reicht ein CO$_2$-Preis?\\ 
{\color{darkgreen}{\Rectangle}}  \fullcite{Creutzig2018}\\ 
{\color{darkgreen}{\Rectangle}}   \fullcite{fabre2025} \\ 
\medskip  \\	

\textsc{Sitzung 5 \dotfill Typologien von Politiken}  \smallskip \\ 
{\color{darkgreen}{\Rectangle}} Welche Politiken hat die Schweiz? Welche Politiken bräuchte sie, hat sie aber nicht? \\
{\color{darkgreen}{\Rectangle}} \fullcite{YanguasParra2025}\\
{\color{darkgreen}{\Rectangle}} \fullcite{Goessling2021}\\
\medskip  \\	

\textsc{Sitzung 6 \dotfill Vorstellung Videokonzepte} \smallskip \\ 
{\color{darkgreen}{\Rectangle}} Einführung in Peer-Review, Review der Videokonzepte.\\
\medskip  \\	


\textsc{Sitzung 7 \dotfill Politikmixe} \smallskip \\ 
{\color{darkgreen}{\Rectangle}} Warum braucht es Politikmixe zur Dekarbonisierung, am Beispiel des des Strassenverkehrs? \\ 
{\color{darkgreen}{\Rectangle}} \fullcite{Axsen2020} \\  
{\color{darkgreen}{\Rectangle}} \fullcite{Mattioli2020} \\  
\medskip  \\	


\textsc{Sitzung 8 \dotfill Gegen Politiken}  \smallskip \\ 
{\color{darkgreen}{\Rectangle}}  Warum sind die Politiken nicht eingeführt? Reicht das Pariser Klimaabkommen nicht? \\%(Quasi-)Experimente \\
{\color{darkgreen}{\Rectangle}} \fullcite{Gazmararian2025} \\
{\color{darkgreen}{\Rectangle}} \fullcite{Bosetti2025} \\  
\medskip  \\	


\textsc{Sitzung 9 \dotfill (Falsch-)Informationen}  \smallskip \\ 
{\color{darkgreen}{\Rectangle}} Herrscht noch ein Informationsdefizit? Was braucht es noch? \\
{\color{darkgreen}{\Rectangle}}  \fullcite{Benegal2025}\\
{\color{darkgreen}{\Rectangle}}  \fullcite{Carnes_Henderson_2025}\\
{\color{darkgreen}{\Rectangle}}  \fullcite{bruckmann2022actualadoption}\\
\medskip  \\	

\textsc{Sitzung 10 \dotfill Föderalismus} \smallskip \\ 
{\color{darkgreen}{\Rectangle}} Gastseminar: Die Rolle des Föderalismus in der Energiepolitik\\
{\color{darkgreen}{\Rectangle}} \fullcite{Dardanelli2018}; \\ {\color{darkgreen}{\Rectangle}} Arbeitspapier von R. Freiburghaus, J. Schmid und I. Stadelmann-Steffen\\
\medskip  \\	

\textsc{Sitzung 11 \dotfill Exkursion} \smallskip \\ 
{\color{darkgreen}{\Rectangle}} Sustainability Science Forum \url{https://sustainability-science-forum.ch/en} \\
{\color{darkgreen}{\Rectangle}} \fullcite{Bruckmann2024SolarTenantsSurvey} \\
{\color{darkgreen}{\Rectangle}} Arbeitspapier von G. Brückmann, A. Torné und I. Stadelmann-Steffen \textit{Stricter regulatory measures for climate mitigation in the residential building sector — distributional concerns and effectiveness considerations for public support}\\
\medskip  \\	

\textsc{Sitzung 12 \dotfill Vorstellung Forschungskonzepte} \smallskip \\ 
{\color{darkgreen}{\Rectangle}} Die Rolle von offener und replizierbarer Forschung; Peer Review der Forschungskonzepte\\
% {\color{darkgreen}{\Rectangle}}...
\medskip  \\	


\textsc{Sitzung 13 \dotfill Fragestunde} \smallskip \\ 
{\color{darkgreen}{\Rectangle}} Beseitigung von Unklarheiten. Puffer. \\  \medskip  \\	

\textsc{Sitzung 14 \dotfill Synthese}  \smallskip \\ 
{\color{darkgreen}{\Rectangle}} Seminarabschluss. Veröffentlichung der Videos.\\
% {\color{darkgreen}{\Rectangle}}...
\medskip  \\	


\noindent\textbf{Bewertung}
Dieses Seminar wird mittels Mitarbeit während der Sitzungen, inklusive der Gruppenarbeit am Video (30\,\%) und peer-review Reports, und der Forschungsarbeit (50\,\%) benotet. Wie üblich, werden 60\,\% der Punkte für das Bestehen der Lehrveranstaltung benötigt, damit Sie die \textbf{6 ECTS} erhalten.\\
Genaue Aufgabenstellungen, für die Erstellung des Videos in Gruppenarbeit und für die Forschungsarbeit werden im Seminar und auf \href{https://ilias.unibe.ch/goto_ilias3_unibe_crs_3414477.html}{Ilias} bekannt gegeben.\\ 




\medskip
\noindent\textbf{Wichtige Termine: Abgaben auf Ilias}
\begin{center} \begin{minipage}{4.5in}
%\textbf{Abgaben auf Ilias}
\begin{flushleft}
Abgabe Videokonzepte \dotfill 21. 10. 2025 (08:00 Uhr) \\
Peer-review Videokonzepte \dotfill 28. 10. 2025 (08:00 Uhr) \\
Abgabe Konzepte Forschungsarbeit \dotfill 01. 12. 2025 (08:00 Uhr) \\
Peer-review Forschungskonzepte \dotfill 05. 12. 2025 (08:00 Uhr) \\
Video \dotfill 16. 12. 2025 (08:00 Uhr) \\
Abgabe Forschungsarbeit \dotfill 31. 01. 2026 (23:55 Uhr) \\
\end{flushleft}
\end{minipage}
\end{center}
\medskip

%EVTL RELEVANT
%https://www.ipw.unibe.ch/studium/studienbetrieb/pruefungen/index_ger.html
%\noindent\textbf{Weitere Hinweise} Bitte konsultieren Sie für Fragen zur Studien- und Prüfungsorganisation das Institut für Politikwissenachften, siehe zum Beispiel \url{https://www.ipw.unibe.ch/studium/studienbetrieb/pruefungen/index_ger.html}.



{\footnotesize \noindent\textbf{Referenzen}
\fullcite{Graeber2024, Zabel2025}}



 \insert\footins{\footnotesize \noindent\textbf{Credits:}
Dieser Syllabus basiert auf einem Template von Harish Guda \url{https://github.com/harish-guda/teaching-resources/blob/master/syllabus-template.pdf}. Ich möchte an dieser Stelle auch allen danken, die mich bei der Erarbeitung dieses Seminars für das HS2025 unterstützt haben, insbesondere: Isabelle Stadelmann-Steffen, Jana Föcker und Mirco Good. Ein besonderer Dank geht an Jonas Schmid für den Gastvortrag.}



%%%%%% THE END
\end{document} 